%% Generated by Sphinx.
\def\sphinxdocclass{report}
\documentclass[a4paper,10pt,english]{sphinxmanual}
\ifdefined\pdfpxdimen
   \let\sphinxpxdimen\pdfpxdimen\else\newdimen\sphinxpxdimen
\fi \sphinxpxdimen=.75bp\relax
\ifdefined\pdfimageresolution
    \pdfimageresolution= \numexpr \dimexpr1in\relax/\sphinxpxdimen\relax
\fi
%% let collapsible pdf bookmarks panel have high depth per default
\PassOptionsToPackage{bookmarksdepth=5}{hyperref}

\PassOptionsToPackage{warn}{textcomp}
\usepackage[utf8]{inputenc}
\ifdefined\DeclareUnicodeCharacter
% support both utf8 and utf8x syntaxes
  \ifdefined\DeclareUnicodeCharacterAsOptional
    \def\sphinxDUC#1{\DeclareUnicodeCharacter{"#1}}
  \else
    \let\sphinxDUC\DeclareUnicodeCharacter
  \fi
  \sphinxDUC{00A0}{\nobreakspace}
  \sphinxDUC{2500}{\sphinxunichar{2500}}
  \sphinxDUC{2502}{\sphinxunichar{2502}}
  \sphinxDUC{2514}{\sphinxunichar{2514}}
  \sphinxDUC{251C}{\sphinxunichar{251C}}
  \sphinxDUC{2572}{\textbackslash}
\fi
\usepackage{cmap}
\usepackage[T1]{fontenc}
\usepackage{amsmath,amssymb,amstext}
\usepackage{babel}



\usepackage{tgtermes}
\usepackage{tgheros}
\renewcommand{\ttdefault}{txtt}



\usepackage[Bjarne]{fncychap}
\usepackage[,numfigreset=1,mathnumfig]{sphinx}

\fvset{fontsize=auto}
\usepackage{geometry}


% Include hyperref last.
\usepackage{hyperref}
% Fix anchor placement for figures with captions.
\usepackage{hypcap}% it must be loaded after hyperref.
% Set up styles of URL: it should be placed after hyperref.
\urlstyle{same}

\addto\captionsenglish{\renewcommand{\contentsname}{Contents:}}

\usepackage{sphinxmessages}
\setcounter{tocdepth}{1}


        \usepackage{graphicx}
        \usepackage{background}

        \backgroundsetup{
          scale=1,
          color=black,
          opacity=1,
          angle=0,
          position=current page.north,
          contents={%
          \small\sffamily%
          \begin{minipage}{.22\textwidth}
          \vspace{1.65cm}
          \hspace{-0.175cm}
          \includegraphics[width=\linewidth,height=70pt,keepaspectratio]{../../images/eumetsat.png}
          \end{minipage}%
          \begin{minipage}{.8\textwidth}
          \vspace{2cm}
          \parbox[b]{.6\textwidth}{}\hfill \\
          \end{minipage}%
          }
        }

        \usepackage{fancyhdr}
        \pagestyle{fancy}
        \fancypagestyle{normal}{%
        \fancyhead{}
        \fancyhead[RE,RO]{\bf{project-docs/solution-design \\ , \today \\ EUMETSAT WP FCIDECOMP - Solution design}}
        \renewcommand{\headrulewidth}{0.5pt}
        \fancyfoot{}
        \fancyfoot[C]{\thepage}
        }
        \fancypagestyle{plain}{%
        \fancyhead{}
        \fancyhead[RE,RO]{\bf{project-docs/solution-design \\ , \today \\ EUMETSAT WP FCIDECOMP - Solution design}}
        \fancyfoot[CO,CE]{\thepage}
        }

        

\title{EUMETSAT WP FCIDECOMP - Solution design}
\date{Nov 10, 2021}
\release{}
\author{EUMETSAT}
\newcommand{\sphinxlogo}{\vbox{}}
\renewcommand{\releasename}{}
\makeindex
\begin{document}

\pagestyle{empty}

        \pagenumbering{Roman} %%% to avoid page 1 conflict with actual page 1

        \sphinxmaketitle

        \clearpage
        \pagenumbering{roman}
        \listoftables
        \clearpage
        \pagenumbering{arabic}

        
\pagestyle{plain}
\sphinxtableofcontents
\pagestyle{normal}
\phantomsection\label{\detokenize{index::doc}}



\chapter{Document Information}
\label{\detokenize{document_info:document-information}}\label{\detokenize{document_info::doc}}

\begin{savenotes}\sphinxattablestart
\centering
\begin{tabular}[t]{|*{3}{\X{1}{3}|}}
\hline

\sphinxAtStartPar
ID
&
\sphinxAtStartPar
:
&
\sphinxAtStartPar
fcidecomp/documentation/solution\sphinxhyphen{}design
\\
\hline
\sphinxAtStartPar
Version
&
\sphinxAtStartPar
:
&
\sphinxAtStartPar

\\
\hline
\sphinxAtStartPar
Authors
&
\sphinxAtStartPar
:
&\begin{enumerate}
\sphinxsetlistlabels{\Alph}{enumi}{enumii}{}{.}%
\setcounter{enumi}{12}
\item {} 
\sphinxAtStartPar
Bottaccio (B\sphinxhyphen{}Open Solutions)

\end{enumerate}
\begin{enumerate}
\sphinxsetlistlabels{\Alph}{enumi}{enumii}{}{.}%
\setcounter{enumi}{12}
\item {} 
\sphinxAtStartPar
Cucchi (B\sphinxhyphen{}Open Solutions)

\end{enumerate}
\\
\hline
\end{tabular}
\par
\sphinxattableend\end{savenotes}


\section{Document Change Record}
\label{\detokenize{document_info:document-change-record}}

\begin{savenotes}\sphinxatlongtablestart\begin{longtable}[c]{|\X{15}{100}|\X{15}{100}|\X{10}{100}|\X{60}{100}|}
\sphinxthelongtablecaptionisattop
\caption{Document Change Record\strut}\label{\detokenize{document_info:id1}}\\*[\sphinxlongtablecapskipadjust]
\hline

\endfirsthead

\multicolumn{4}{c}%
{\makebox[0pt]{\sphinxtablecontinued{\tablename\ \thetable{} \textendash{} continued from previous page}}}\\
\hline

\endhead

\hline
\multicolumn{4}{r}{\makebox[0pt][r]{\sphinxtablecontinued{continues on next page}}}\\
\endfoot

\endlastfoot

\sphinxAtStartPar
Issue / Revision
&
\sphinxAtStartPar
Date
&
\sphinxAtStartPar
DCN. No
&
\sphinxAtStartPar
Changed Pages / Paragraphs
\\
\hline
\sphinxAtStartPar

&
\sphinxAtStartPar
15 Nov 2021
&&
\sphinxAtStartPar
First version.
\\
\hline
\end{longtable}\sphinxatlongtableend\end{savenotes}


\chapter{Introduction}
\label{\detokenize{introduction:introduction}}\label{\detokenize{introduction::doc}}

\section{Purpose}
\label{\detokenize{introduction:purpose}}
\sphinxAtStartPar
The document describes a design proposal for a maintainable solution allowing users to reliably decode FCI L1c products
compressed with CharLS.


\section{Reference Documents}
\label{\detokenize{introduction:reference-documents}}

\begin{savenotes}\sphinxatlongtablestart\begin{longtable}[c]{|\X{20}{100}|\X{30}{100}|\X{50}{100}|}
\sphinxthelongtablecaptionisattop
\caption{Reference documents\strut}\label{\detokenize{introduction:id1}}\\*[\sphinxlongtablecapskipadjust]
\hline
\sphinxstyletheadfamily 
\sphinxAtStartPar
\#
&\sphinxstyletheadfamily 
\sphinxAtStartPar
Title
&\sphinxstyletheadfamily 
\sphinxAtStartPar
Reference
\\
\hline
\endfirsthead

\multicolumn{3}{c}%
{\makebox[0pt]{\sphinxtablecontinued{\tablename\ \thetable{} \textendash{} continued from previous page}}}\\
\hline
\sphinxstyletheadfamily 
\sphinxAtStartPar
\#
&\sphinxstyletheadfamily 
\sphinxAtStartPar
Title
&\sphinxstyletheadfamily 
\sphinxAtStartPar
Reference
\\
\hline
\endhead

\hline
\multicolumn{3}{r}{\makebox[0pt][r]{\sphinxtablecontinued{continues on next page}}}\\
\endfoot

\endlastfoot

\sphinxAtStartPar
{[}CONDA\_VARIANTS{]}

\phantomsection\label{\detokenize{introduction:conda-variants}}&
\sphinxAtStartPar
conda\sphinxhyphen{}build \textendash{} Build variants
&
\sphinxAtStartPar
\sphinxurl{https://docs.conda.io/projects/conda-build/en/latest/resources/variants.html}
\\
\hline
\sphinxAtStartPar
{[}FCIDECOMP\_CONDA{]}

\phantomsection\label{\detokenize{introduction:fcidecomp-conda}}&
\sphinxAtStartPar
FCIDECOMP Conda recipe developed by Martin Raspaud (SMHI)
&
\sphinxAtStartPar
\sphinxurl{https://github.com/mraspaud/fcidecomp-conda-recipe/}
\\
\hline
\sphinxAtStartPar
{[}FCIDECOMP\_LATEST{]}

\phantomsection\label{\detokenize{introduction:fcidecomp-latest}}&
\sphinxAtStartPar
FCIDECOMP v1.0.2 repository
&
\sphinxAtStartPar
\sphinxurl{https://sftp.eumetsat.int/public/folder/UsCVknVOOkSyCdgpMimJNQ/User-Materials/Test-Data/MTG/MTG\_FCI\_L1C\_Enhanced-NonN\_TD-272\_May2020/FCI\_Decompression\_Software\_V1.0.2/EUMETSAT-FCIDECOMP\_V1.0.2.tar.gz}
\\
\hline
\sphinxAtStartPar
{[}FCIDECOMP\_TEST\_DATA{]}

\phantomsection\label{\detokenize{introduction:fcidecomp-test-data}}&
\sphinxAtStartPar
FCIDECOMP v1.0.2 test data
&
\sphinxAtStartPar
\sphinxurl{https://sftp.eumetsat.int/public/folder/UsCVknVOOkSyCdgpMimJNQ/User-Materials/Test-Data/MTG/MTG\_FCI\_L1C\_Enhanced-NonN\_TD-272\_May2020/}
\\
\hline
\sphinxAtStartPar
{[}FCIDECOMP\_WPD{]}

\phantomsection\label{\detokenize{introduction:fcidecomp-wpd}}&
\sphinxAtStartPar
Work Package Description
&
\sphinxAtStartPar
EUM/SEP/WPD/21/1244304
\\
\hline
\sphinxAtStartPar
{[}HDF5PLUGIN{]}

\phantomsection\label{\detokenize{introduction:hdf5plugin}}&
\sphinxAtStartPar
\sphinxcode{\sphinxupquote{hdf5plugin}} python package
&
\sphinxAtStartPar
\sphinxurl{https://github.com/silx-kit/hdf5plugin}
\\
\hline
\sphinxAtStartPar
{[}HDFVIEW{]}

\phantomsection\label{\detokenize{introduction:hdfview}}&
\sphinxAtStartPar
HDFView Software
&
\sphinxAtStartPar
\sphinxurl{https://www.hdfgroup.org/downloads/hdfview/}
\\
\hline
\sphinxAtStartPar
{[}MTG4AFRICA{]}

\phantomsection\label{\detokenize{introduction:mtg4africa}}&
\sphinxAtStartPar
EUMETSAT Data Tailor mtg4africa plugin
&
\sphinxAtStartPar
\sphinxurl{https://gitlab.eumetsat.int/data-tailor/support-to-mtg/mtg4africa}
\\
\hline
\sphinxAtStartPar
{[}NETCDF\_C{]}

\phantomsection\label{\detokenize{introduction:netcdf-c}}&
\sphinxAtStartPar
Unidata \sphinxhyphen{} NetCDF\sphinxhyphen{}C
&
\sphinxAtStartPar
\sphinxurl{https://docs.unidata.ucar.edu/netcdf-c/current/}
\\
\hline
\sphinxAtStartPar
{[}NETCDF\_JAVA{]}

\phantomsection\label{\detokenize{introduction:netcdf-java}}&
\sphinxAtStartPar
Unidata \sphinxhyphen{} NetCDF\sphinxhyphen{}Java
&
\sphinxAtStartPar
\sphinxurl{https://www.unidata.ucar.edu/software/netcdf-java/}
\\
\hline
\sphinxAtStartPar
{[}NETCDF\_JAVA\_GITHUB{]}

\phantomsection\label{\detokenize{introduction:netcdf-java-github}}&
\sphinxAtStartPar
NetCDF\sphinxhyphen{}C for reading (nj22Config.xml) in non\sphinxhyphen{}Unidata netCDF\sphinxhyphen{}Java based tools
&
\sphinxAtStartPar
\sphinxurl{https://github.com/Unidata/thredds/issues/1063}
\\
\hline
\sphinxAtStartPar
{[}PANOPLY{]}

\phantomsection\label{\detokenize{introduction:panoply}}&
\sphinxAtStartPar
Panoply netCDF, HDF and GRIB Data Viewer
&
\sphinxAtStartPar
\sphinxurl{https://www.giss.nasa.gov/tools/panoply/}
\\
\hline
\end{longtable}\sphinxatlongtableend\end{savenotes}


\chapter{Creation of canonical repository}
\label{\detokenize{canonical_repository:creation-of-canonical-repository}}\label{\detokenize{canonical_repository::doc}}

\section{Introduction}
\label{\detokenize{canonical_repository:introduction}}
\sphinxAtStartPar
A canonical repository is established on the EUMETSAT GitLab service at \sphinxurl{https://gitlab.eumetsat.int/sepdssme/fcidecomp}
for development purposes. Each time a new release is produced, the corresponding code is synchronized to the public
EUMETSAT Open Source repository at \sphinxstylestrong{XXX {[}NOTE: abbiamo una reference?{]}}.


\section{Repository initialization}
\label{\detokenize{canonical_repository:repository-initialization}}\label{\detokenize{canonical_repository:id1}}
\sphinxAtStartPar
{\hyperref[\detokenize{introduction:fcidecomp-latest}]{\sphinxcrossref{\DUrole{std,std-ref}{FCIDECOMP v1.0.2}}}} is taken as blueprint for the development of the solution codebase.

\sphinxAtStartPar
The repository is put under configuration control. A new minor release adding README, BUILD, INSTALL, and LICENCE
files, starting the Changelog, codifying the use of semantic versioning for future versions and adding a standardised
build system is published.


\section{Test suite}
\label{\detokenize{canonical_repository:test-suite}}
\sphinxAtStartPar
An initial test suite (at least against nominal conditions) is implemented following the V\&V strategy defined in
\sphinxstylestrong{{[}TODO: add reference{]}}. Most tests are implemented as automated tests against the Python interface.


\section{Test data}
\label{\detokenize{canonical_repository:test-data}}
\sphinxAtStartPar
A preliminary set of test data taken from the {\hyperref[\detokenize{introduction:fcidecomp-test-data}]{\sphinxcrossref{\DUrole{std,std-ref}{MTG FCI L1C test data}}}} is added to ensure a
consistent and permanent dataset to execute tests.


\chapter{Support to required usage patterns}
\label{\detokenize{support_to_usage_patterns:support-to-required-usage-patterns}}\label{\detokenize{support_to_usage_patterns::doc}}

\section{Introduction}
\label{\detokenize{support_to_usage_patterns:introduction}}
\sphinxAtStartPar
This section describes the strategies adopted to ensure that the FCIDECOMP software supports the required usage
patterns.

\sphinxAtStartPar
\sphinxstylestrong{{[}NOTA: questo l’ho spostato qui da Packaging and deployment{]}}

\sphinxAtStartPar
As a baseline, the FCIDEOCOMP software supports HDF5 1.10. Strategies to grant support for multiple versions of HDF5
described in the \sphinxtitleref{Further developments appendix \textless{}further\_developments}.


\section{Integration with tools based on netCDF\sphinxhyphen{}C}
\label{\detokenize{support_to_usage_patterns:integration-with-tools-based-on-netcdf-c}}
\sphinxAtStartPar
\sphinxstylestrong{{[}NOTA: mi sembra che questa parte stia meglio in un ulteriore sottoparagrafo invece che nell’intro,
ma se non sei d’accordo la sposto su e tolgo questo paragrafo{]}}

\sphinxAtStartPar
The current implementation of the FCIDECOMP software ({\hyperref[\detokenize{introduction:fcidecomp-latest}]{\sphinxcrossref{\DUrole{std,std-ref}{v1.0.2}}}}) which, as mentioned in the
{\hyperref[\detokenize{canonical_repository:repository-initialization}]{\sphinxcrossref{\DUrole{std,std-ref}{Repository initialization}}}} paragraph serves as blueprint for the software codebase,
already satisfies the HDF5 filters interface. Given this, integration with utilities relying on the \sphinxcode{\sphinxupquote{netcdf\sphinxhyphen{}c}}
library ({\hyperref[\detokenize{introduction:netcdf-c}]{\sphinxcrossref{\DUrole{std,std-ref}{{[}NETCDF\sphinxhyphen{}C{]}}}}}) is ensured, provided that:
\begin{itemize}
\item {} 
\sphinxAtStartPar
the location of the FCIDECOMP filter library is specified in a specific environment variable, \sphinxcode{\sphinxupquote{HDF5\_PLUGIN\_PATH}};

\item {} 
\sphinxAtStartPar
the correct filter id (32018 for FCIDECOMP), if required by the utility, is specified;

\end{itemize}


\section{Usage as CLI tool}
\label{\detokenize{support_to_usage_patterns:usage-as-cli-tool}}
\sphinxAtStartPar
In order to provide a baseline support for CLI usage of the FCIDECOMP software, \sphinxcode{\sphinxupquote{nccopy}} (software utility of the
\sphinxcode{\sphinxupquote{netcdf\sphinxhyphen{}c}} library) is chosen as reference standard CLI tool. To foster integration with \sphinxcode{\sphinxupquote{nccopy}}, the FCIDECOMP
software provides to:
\begin{itemize}
\item {} 
\sphinxAtStartPar
put the filter’s library to a specific path at installation

\item {} 
\sphinxAtStartPar
have the \sphinxcode{\sphinxupquote{HDF5\_PLUGIN\_PATH}} environment variable automatically set each time a conda environment where FCIDECOMP is installed get activated

\end{itemize}

\sphinxAtStartPar
The FCIDECOMP software documentation also provides instructions on how to call \sphinxcode{\sphinxupquote{nccopy}} to decompress files using the
FCIDECOMP filter.


\section{Integration with Python}
\label{\detokenize{support_to_usage_patterns:integration-with-python}}
\sphinxAtStartPar
Integration with Python is provided by a small Python package developed ad hoc, which satisfies the required \sphinxcode{\sphinxupquote{h5py}}
interface to make the FCIDECOMP filter available for Python applications. Such package, based upon a stripped\sphinxhyphen{}down
version of the {\hyperref[\detokenize{introduction:hdf5plugin}]{\sphinxcrossref{\DUrole{std,std-ref}{hdf5plugin}}}} package, is essentially composed of an \sphinxcode{\sphinxupquote{\_\_init\_\_.py}} defining the
filter interface to \sphinxcode{\sphinxupquote{h5py}}.

\sphinxAtStartPar
See the {\hyperref[\detokenize{a_integration_with_hdf5plugin:integration-with-hdf5plugin}]{\sphinxcrossref{\DUrole{std,std-ref}{Integration with hdf5plugin appendix}}}} for details on the integration with the widely used {\hyperref[\detokenize{introduction:hdf5plugin}]{\sphinxcrossref{\DUrole{std,std-ref}{hdf5plugin}}}} package and interation
with its maintainers’ community.


\section{Integration with EUMETSAT Data Tailor}
\label{\detokenize{support_to_usage_patterns:integration-with-eumetsat-data-tailor}}
\sphinxAtStartPar
At the moment, the Data Tailor supports reading compressed FCI L1C products through the optional
\sphinxcode{\sphinxupquote{epct\_plugin\_mtg4africa}} {\hyperref[\detokenize{introduction:mtg4africa}]{\sphinxcrossref{\DUrole{std,std-ref}{customisation plugin}}}}, which in turns install FCIDECOMP by installing
with \sphinxcode{\sphinxupquote{pip}} the \sphinxcode{\sphinxupquote{hdf5plugin}} package.

\sphinxAtStartPar
The approach to integrate the described solution with the Data Tailor includes a revision of the current
build and installation approach for the \sphinxcode{\sphinxupquote{epct\_plugin\_mtg4africa}} customisation plugin, so that it
installs the FCIDECOMP support through the Python package described above and its dependencies.


\section{Integration with tools based on netCDF\sphinxhyphen{}Java}
\label{\detokenize{support_to_usage_patterns:integration-with-tools-based-on-netcdf-java}}
\sphinxAtStartPar
{\hyperref[\detokenize{introduction:panoply}]{\sphinxcrossref{\DUrole{std,std-ref}{Panoply}}}} and {\hyperref[\detokenize{introduction:hdfview}]{\sphinxcrossref{\DUrole{std,std-ref}{HDFView}}}} have been identified as the key software based on netCDF\sphinxhyphen{}Java
to support. The integration of the FCIDECOMP software in these applications can be achieved by instructing them
to use the netCDF\sphinxhyphen{}C library (instead of netCDF\sphinxhyphen{}Java) to read netCDF files
(see related {\hyperref[\detokenize{introduction:netcdf-java-github}]{\sphinxcrossref{\DUrole{std,std-ref}{github issue}}}}). Support is then granted by describing the aforementioned
procedure in the FCIDECOMP software documentation.

\sphinxAtStartPar
The issue of a generic integration with {\hyperref[\detokenize{introduction:netcdf-java}]{\sphinxcrossref{\DUrole{std,std-ref}{Unidata Netcdf\sphinxhyphen{}Java}}}} is discussed in the
{\hyperref[\detokenize{a_design_justifications:design-justifications}]{\sphinxcrossref{\DUrole{std,std-ref}{Design justification appendix}}}}.


\chapter{Packaging and deployment}
\label{\detokenize{packaging_and_deployment:packaging-and-deployment}}\label{\detokenize{packaging_and_deployment::doc}}

\section{Introduction}
\label{\detokenize{packaging_and_deployment:introduction}}
\sphinxAtStartPar
In the following paragraphs the strategy to build and package the FCIDECOMP software in order to ensure
support for all the required systems is reported.


\section{Supported platforms}
\label{\detokenize{packaging_and_deployment:supported-platforms}}\label{\detokenize{packaging_and_deployment:id1}}
\sphinxAtStartPar
The FCIDECOMP software supports the following platforms:
\begin{itemize}
\item {} 
\sphinxAtStartPar
Windows 10, 32 and 64 bit

\item {} 
\sphinxAtStartPar
Ubuntu 18.04, Ubuntu 20.04 64 bit

\item {} 
\sphinxAtStartPar
CentOS 7 64 bit

\end{itemize}

\sphinxAtStartPar
Details on the selection process leading to the list presented above are provided in the
{\hyperref[\detokenize{a_design_justifications:design-justifications}]{\sphinxcrossref{\DUrole{std,std-ref}{Design justification appendix}}}}.


\section{Building the binaries}
\label{\detokenize{packaging_and_deployment:building-the-binaries}}
\sphinxAtStartPar
The build system for the software binaries is drawn from the one used in the
{\hyperref[\detokenize{introduction:fcidecomp-latest}]{\sphinxcrossref{\DUrole{std,std-ref}{FCIDECOMP v1.0.2 source code}}}}, and adapted from there to guarantee support for all the
required systems.


\section{Packaging as a Conda package}
\label{\detokenize{packaging_and_deployment:packaging-as-a-conda-package}}
\sphinxAtStartPar
Packages are built using Conda, as it provides standardised environments with a large set of pre\sphinxhyphen{}compiled packages.
From the point of view of Conda, the operating systems listed in the {\hyperref[\detokenize{packaging_and_deployment:supported-platforms}]{\sphinxcrossref{\DUrole{std,std-ref}{Supported platforms}}}}
paragraph can be considered as two groups of OS: in Conda standardised environment it is enough to build the package for
one Linux distribution in order to make it compatible with other Linux distributions. So two conda packages are
released: one for Linux distributions, and one for Windows 10.

\sphinxAtStartPar
These conda packages install both the FCIDECOMP libraries and its Python bindings. As a blueprint for the
conda recipes, the {\hyperref[\detokenize{introduction:fcidecomp-conda}]{\sphinxcrossref{\DUrole{std,std-ref}{Conda recipe}}}} for the packaging of FCIDECOMP mantained by Martin Raspaud
from the Swedish Meteorological and Hydrological Institute has been used.

\sphinxAtStartPar
Conda packages are uploaded to EUMETSAT Anaconda repository \sphinxstylestrong{{[}NOTA: abbiamo una reference?{]}}.


\section{Packaging process}
\label{\detokenize{packaging_and_deployment:packaging-process}}
\sphinxAtStartPar
GitLab CI/CD pipelines to compile, build, test and upload the conda packages to EUMETSAT Anaconda repository are
implemented.

\sphinxAtStartPar
Two GitLab runners are implemented, one with a Docker executor on Linux and the other with a Shell executor on Windows.
\sphinxstylestrong{{[}NOTA: se va presentata come una cosa già fatta, come inserire che non siamo sicure se serva un altro runner per Windows 32\sphinxhyphen{}bit?{]}}


\chapter{Appendix \sphinxhyphen{} Traceability matrix}
\label{\detokenize{a_traceability_matrix:appendix-traceability-matrix}}\label{\detokenize{a_traceability_matrix::doc}}

\chapter{Appendix \sphinxhyphen{} List of users and developers currently using FCIDECOMP}
\label{\detokenize{a_users_using_fcidecomp:appendix-list-of-users-and-developers-currently-using-fcidecomp}}\label{\detokenize{a_users_using_fcidecomp::doc}}

\chapter{Appendix \sphinxhyphen{} Design justifications}
\label{\detokenize{a_design_justifications:appendix-design-justifications}}\label{\detokenize{a_design_justifications:design-justifications}}\label{\detokenize{a_design_justifications::doc}}

\chapter{Appendix \sphinxhyphen{} Long\sphinxhyphen{}term preservation of dependencies}
\label{\detokenize{a_preservation_of_dependencies:appendix-long-term-preservation-of-dependencies}}\label{\detokenize{a_preservation_of_dependencies::doc}}

\chapter{Appendix \sphinxhyphen{} Integration with \sphinxstyleliteralintitle{\sphinxupquote{hdf5plugin}}}
\label{\detokenize{a_integration_with_hdf5plugin:appendix-integration-with-hdf5plugin}}\label{\detokenize{a_integration_with_hdf5plugin:integration-with-hdf5plugin}}\label{\detokenize{a_integration_with_hdf5plugin::doc}}

\chapter{Appendix \sphinxhyphen{} Further developments}
\label{\detokenize{a_further_developments:appendix-further-developments}}\label{\detokenize{a_further_developments:further-developments}}\label{\detokenize{a_further_developments::doc}}


\renewcommand{\indexname}{Index}
\printindex
\end{document}